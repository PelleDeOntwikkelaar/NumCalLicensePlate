\documentclass[main.tex]{subfiles}

\section{Problem Introduction}
The general information about the project reported in this documented is presented in this section.
\subsection{Description}
In recent years or decade, a new way of traffic/speed control is introduced. 
The so called trajectory control \cite{dxcTech}. 
This way of traffic control exists in monitoring traffic at entry and exit points of a given trisect.
With monitoring is meant the identification of vehicles. 
In this case cameras are used with a computer system that recognizes license plates.
By comparing times at entry and exit point an average speed can be calculated. 
In general this method of speed control is better than normal one point control due to the avoidance of the break-and-accelerate syndrome.
This system can be used for a lot more than just speed control. 
By analyzing data, conclusion can be drawn why people change lanes, etc. 
Maybe it will become possible to predict traffic jams, accidents etc. 
Both present systems and futuristic extension relay on the basic principle of license plate recognition from images.
This is a very interesting and also fundamental subject in the modern world.

\subsection{Matlab functionality}
The recognizing of license plates is no more than an Image processing problem.
This is possible with Matlab because in digital terms a photo is a matrix of values and can be analyzed.\cite{rgbWiki}\cite{imreadMat}
Matrix manipulation is very straight forward in Matlab, you could say Matlab is build for Matrix manipulation.
In this way is this a perfect example as a Matlab project.

\subsection{Goals and Objectives}
For a project of any size it is very important to clearly define the goals.
This gives a clear view off direction, whether the project consists of research, development or even experiments.
\begin{itemize}
    \item Creating a script that manipulates the images so the license plate becomes clear.
    \item Creating a script that returns the license plate when (in string format) when a picture is given as input.
    \item Provided enough test data and results to confirm all stated conclusions.
\end{itemize}
\subsection{Possible Extensions}
Further in this report is a larger section devoted to possible extensions, this is only a foretaste.
\subsubsection{Moving Images}
In real life this system works with cameras so with image processing on frames.
An extension can be giving an input of video files instead of just pictures.
\subsubsection{CUDA}
CUDA is a programming language built on top of C that lets the user control the Graphic processor.\cite{nvidiaCuda}
Because image processing contains a lot of parallel calculations, it can be interesting to see what the gain is when executing on a GPU.\cite{matCUDA}