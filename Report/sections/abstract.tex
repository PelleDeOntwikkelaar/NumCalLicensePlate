\documentclass[main.tex]{subfiles}
Matlab is a key example of software used by every engineer, no matter the profession or sector.
The versatility of possibilities is so larger that this software is used from pure mathematical calculations up to chemical simulations.
This project is perfect to illustrate that fact.
The main objective of this project is designing a system that can recognize a license plate from a picture of a car.
As a theoretical example it makes a very good bridge to the real world because systems for security and traffic control make constant use of these kind of software.
When coupling it back from the real world engineering example to the theoretical, we can see a wide range of interests.
The first and most obvious one is the ability to calculate a lot of mathematical problems with software and write custom made functions.
More important is the image manipulation what is in fact a certain way of signal processing.
By working with image manipulation a lot is learned about how computer systems store and use images.
\par
The objective of recognizing license plates seemed through the process not that easy as first thought.
Although it is not difficult to design a system that can recognize certain things in a picture, it is much more complex when these things are not predefined.
There is a classic saying referring to this matter: "Although a computer can calculate a thousand ways to defeat you in a chess game before you know your next move, it cannot distinguish an apple from a banana".
This saying is completely out dated because with the entrance of machine learning and AI, computers can for a while now detect almost everything.
But my project is not based on AI or machine learning algorithms and keeps 100\% in the world of pure mathematical calculations, therefor the saying is perfectly applicable to my project.
\par
As a result the system described in this report is perfectly capable of recognizing a license plate when following conditions are met:
The license plate has to be clearly visible, and the characteristics have to be pre known.\footnote{With characteristics is meant how many characters and how many of those are letters.}
The system is not capable of detecting all license plates without pre knowledge.
The system does only accept images as input, not a camera with moving images and the system make use of the CPU, not of parallel processing.
\par
This makes it possible to state the following conclusions:
The system does for fill it's purpose but the system can be used as a starting point for a better and more general system.
