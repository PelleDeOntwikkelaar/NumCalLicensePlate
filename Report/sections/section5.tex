\documentclass[main.tex]{subfiles}
\section{Development process}
In this section the development process is discussed. 
Before getting into the details, first a short description on why this is important.
Although \textit{Matlab} is a scientific software tool, making projects with \textit{Matlab} can be compared to a software development process.
In software development processes, a very important part is planning and way of developing.
The way of developing has a direct impact on the chances to succeed, or chances to meet the objectives.
Personally I have some experience with (small) software projects so I started this project with my experience in mind.
\par
The first and very import stage is the preparation and planning.
This is a big stage in the project because the better this is done, the easier it is to make and develop the project.
With better is not meant more into detail, with better realistic and thoroughly are the real goals.
The proper way of planning can easily be found in section 2 Problem Breakdown. 
There a dividends in different parts with each own sub problems is clearly visible.
This way of working is necessary to keep track of the complete project and state clear objective goals.
\par
After reading a lot online and in a very interesting book: TODO link image manipulation book, I fastly came to the conclusion that the image manipulation was the best way to start working on this project.
In the case of libraries and concepts, this was also the most new part for me.
Before working on this project, my matrix manipulation mostly concentrated on signal processing or audio filtering. 
Without having the exact necessary knowledge, it was still possible to define an end goal for this part, this because the end goal is straight forward: An image with all parts detected.
\par
The next part, referred to as the search, is not so much based on new knowledge but more implementation of my personal ideas how this would work.
By defining the sub problems as stated, it gives the developer (me) clear small objectives, what is most of all important in parts were the projects builds on own knowledge.
\par
The same planning tactic can also be used for writing the report. 
Write objectives for your report: list all necessary items, design a template, make the base structure.
Then write the first version, puzzled into the designed structure.
\par
To keep track of changes and as a safety procedure, I did put the project (code and report) on \textit{GitHub}. 
Working with a Git server not only provides a back up but the luxury of going back in time when certain mistakes are made.
