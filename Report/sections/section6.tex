\documentclass[main.tex]{subfiles}
\section{Critical Reflection}
To wright a critical reflection it is important to stretch the difference between a critical reflection and a conclusion.
The conclusion will focus on the stated objectives and the results.
An evaluation is made and in this evaluation also is room for a critical session.
However the conclusion mostly focuses on the beginning and ending.
\par
In this section, the critical reflection, the whole process is discussed en evaluated.
From making the objectives to the developing process to failures or non met objectives.
\subsection{The objectives}
The objectives of this project were straight forward: recognize the license plates based on an input picture.
To this objective a few other cases were added.
Look into the possibility of moving images and CUDA (parallel processing).
When looking critical to these objectives it is safe to say that there didn't went a lot of thinking in it.
The main objective is clear and also realistic.
But the objective of looking into moving images isn't.
This objective was stated without a lot of pre knowledge in image processing and especially working with film.
If I had done my research before starting the development of this project. I would have made the conclusion that this is a very difficult extension.
This extension is probably a complete project on his own. 
More about the what and why in the section "possible extensions".
\par
The second secondary objective was a good choice to state. 
Unfortunately the CUDA development didn't work out but it was a good thing to state it as an extra objective.
If the original scope of the project was a little different it might have been possible to start with parallel processing but this wasn't the case.
In my scope I developed a program that is machine independent. 
This will mean it will work on any machine that can execute \textit{Matlab/Octave} code.
This makes it very difficult to start with parallel processing because this depends on the GPU and direct control of the GPU is different on every machine combination. \footnote{Machine combination means operation system in combination with type of GPU.}
For an example on a \textit{Linux} operating system with a NVidia CUDA enabled processor it is very easy to start developing parallel processing, keeping in mind that on Linux octave is used instead of Matlab.
\subsection{Development process}
To reflect the development process it is necessary to introduce a time scope.
The start of the development lays in February after the introduction of the objective.
The finish and most of the reporting came after Semana Santa. 
This wide spread time period with the combination of other project and causes results in a lot of wasted time.
Although the project is not very complex, it also not straight forward. 
This results in a start-up and work-in time needed after every break in development.
Fortunately I already had the pre knowledge that I described in the previous section, this helped me to make a strategic plan and minimize the loss of time.
This problems and loss of time had in no way to do with the way this course is designed or scheduled, it is only a result of the combination of the different works and dividing of my own attention between different causes.
\par
The research done for the different part went very smooth because I was in the possession of a very good book on the matter.\cite{gonzalez2008digital}
Also online are a lot of examples of image processing and container regions available.
At the end the most difficult job was connecting the dots.