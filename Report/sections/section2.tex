\documentclass[main.tex]{subfiles}
\section{Problem Breakdown}
In this section the general problem is brought down to simpilar and smaller sub problems.
The goals of this section is not making more objectives but making a clear and good dividence in the project.
Important in breaking down a problem is th clear definition off the sub problems.
It has to be very clear what every part consists off and most important when a sub problem is solved.
Out off this last defintion follows that it has to be possible to test every different sub problem in a convenient way.
To summarize: a project consists off different sub problems which all have the following.
\begin{itemize}
    \item Title
    \item Clear and simple objective.
    \item Orientation whitin the whole project.
    \item Test conditions.
\end{itemize}
Multiple sub problems can be combined into a project phase.
When talking about project phases, the following phases can be defined, there is a large resemblance with the goals defind in the previous section.
\begin{itemize}
    \item Image manipulation: Manipulate the input image to a form where it is easy to start the proces of finding the license plate.
    \item The search: The search off the interire (manipulated) image to the license plate.
    \item Evaluation: Confirmation or denial. 
\end{itemize}
These project phases each consist of multiple sub problems, these are defined in the following subsections.
\subsection{Image manipulation}
As earlier described consists this project phase of preparing the image for the search off a license plate.
The preparation of this image consists of multiple steps.
Each of these steps can be considerd a sub problem.
\begin{enumerate}
    \item \label{sub:uniform} \textbf{Uniform:} Transform the image to a uniform dimension. In this way all images from this point on have accactly the same characteristics.  
    \item \label{sub:greyscale} \textbf{Greyscale:} Transform the image to a greyscale version off itself.
    \item \label{sub:noise} \textbf{Noise:} Removal of possible noise in the picture.
    \item \label{sub:detectAllEdges} \textbf{Edges: }Detecting off all the possible edges in an image.
    \item \label{sub:clearEdges} \textbf{Clear edges: }Clear the image to make detected lines the only visible things.
    \item \label{sub:fillRegions} \textbf{Fill edges: }Detect non straight edges and fill them.
\end{enumerate}
\subsection{The Search}
Before getting into the search, first a few words of explenation about the divide between the image manipulation and search process.
We live in a world were a lot of things are the same in the whole world, a few of them are: the use of license plates, the need for speed control, and things in these categories.
When looking at license plates, although the concept of a license plate seems universal, it is easy to recognize that they are different around the world.
Some of them have only letters, others have two letter and four numbers, etc.
Even inside the european union there are a lot of differences. 
When building a license plate recognizer it is clear to see that a big part of the process will not change: this is the regnizing of edges and filling possible areas.
This will always result in an image with intersting areas coloured in, no matter the composition of the licinseplate to be found.
It is because of this fact that I divided search and image manipulation.
No matter which (country) licenseplate that has to be found. The image manipulation part will always be the same.
\begin{enumerate}
    \item 
\end{enumerate}


\subsection{Evaluation}
