%%%%%%%%%%%%%%%%%%%%%%%%%%%%%%%%%%%%%%%%%
% LaTeX Template
% Contributors:
% - Gilles Callebaut
% - Guus Leenders
% - Stijn Crul
% - Pelle Reyniers
% - Wouter Legiest
%
% Original author:
% Linux and Unix Users Group at Virginia Tech Wiki
% (https://vtluug.org/wiki/Example_LaTeX_chem_lab_report)
%
% License:
% CC BY-NC-SA 3.0 (http://creativecommons.org/licenses/by-nc-sa/3.0/)
%
%%%%%%%%%%%%%%%%%%%%%%%%%%%%%%%%%%%%%%%%%

%----------------------------------------------------------------------------------------
%	PACKAGES AND DOCUMENT CONFIGURATIONS
%----------------------------------------------------------------------------------------

\documentclass[twoside,a4paper]{article}
    \usepackage{geometry}
    
    \usepackage{siunitx} % Provides the \SI{}{} and \si{} command for typesetting SI units
    \usepackage{graphicx} % Required for the inclusion of images
    %\usepackage{natbib} % Required to change bibliography style to APA
    \usepackage{amsmath} % Required for some math elements
    \usepackage{lastpage} % know last page (used in fancy header)
    \usepackage{babel}
    \usepackage{float}
    %\usepackage{cite}
    \usepackage[usenames,dvipsnames]{xcolor}
    %\setcitestyle{numbers,super}
    \usepackage{fancyhdr} % Fancy Header
    \usepackage[xindy, toc, numberedsection]{glossaries} % glossaries with xindy (recommended) for the indexing phase and show glossaries in TOC, and numberedsection to get a Setion number in the title
    \usepackage{url} %The command is intended for email addresses, hypertext links, directories/paths, etc., which normally have no spaces, so by default the package ignores spaces in its argument.
    \usepackage{listings} 
    \usepackage{subfiles}
    \graphicspath{{images/}{../images/}}
    % it allows formatting and highlighting source code
    % in addition, Pygments must be installed
    % How to install on Windows:
    % 1) install python (and add it to your PATH)
    % 2) install pip (https://pip.pypa.io/en/stable/installing/)
    % 3) install pygments (pip install Pygments)
    % add pygments to your PATH, the command "pip show Pygments" shows where the lib is installed
    % Of course, pdfLaTeX (or whatever engine/format you use) still has to be called with the -shell-escape option.
    
    \usepackage{varioref}
    
    %\loadglsentries{glossaries.tex}
    %\makeglossaries % generate the glossary
    % Any links in resulting glossary will not be "clickable" unless you load the glossaries package after the hyperref package.
    
    \setlength\parindent{0pt} % Removes all indentation from paragraphs
    
    %\renewcommand{\labelenumi}{\alph{enumi}.} % Make numbering in the enumerate environment by letter rather than number (e.g. section 6)
    
    \renewcommand{\arraystretch}{1.2} % Increasing the array stretch factor using \renewcommand{\arraystretch}{<factor>} where <factor> is a numeric value
    
    %\usepackage{times} % Uncomment to use the Times New Roman font
    
    % This is the color used for MATLAB comments below
    \definecolor{MyDarkGreen}{rgb}{0.0,0.4,0.0}
    
    % For faster processing, load Matlab syntax for listing
    \lstloadlanguages{Matlab}%
    \lstset{language=Matlab,                        % Use MATLAB
        frame=single,                           % Single frame around code
        basicstyle=\small\ttfamily,             % Use small true type font
        keywordstyle=[1]\color{blue}\bfseries,  % MATLAB functions bold and blue
        keywordstyle=[2]\color{purple},         % MATLAB function arguments purple
        keywordstyle=[3]\color{blue}\underbar,  % User functions underlined and blue
        identifierstyle=,                       % Nothing special about identifiers
                                                % Comments small dark green courier
        commentstyle=\usefont{T1}{pcr}{m}{sl}\color{MyDarkGreen}\small,
        stringstyle=\color{purple},             % Strings are purple
        showstringspaces=false,                 % Don't put marks in string spaces
        tabsize=5,                              % 5 spaces per tab
        %
        %%% Put standard MATLAB functions not included in the default
        %%% language here
        morekeywords={xlim,ylim,var,alpha,factorial,poissrnd,normpdf,normcdf},
        %
        %%% Put MATLAB function parameters here
        morekeywords=[2]{on, off, interp},
        %
        %%% Put user defined functions here
        morekeywords=[3]{FindESS, homework_example},
        %
        morecomment=[l][\color{blue}]{...},     % Line continuation (...) like blue comment
        numbers=left,                           % Line numbers on left
        firstnumber=1,                          % Line numbers start with line 1
        numberstyle=\tiny\color{blue},          % Line numbers are blue
        stepnumber=5,                           % Line numbers go in steps of 5
        breaklines=true
        }

        % Includes a MATLAB script.
        % The first parameter is the label, which also is the name of the script
        %   without the .m.
        % The second parameter is the optional caption.

    \newcommand{\matlabscript}[2]
        {\begin{itemize}\item[]\lstinputlisting[caption=#2,label=#1]{#1.m}\end{itemize}}
    
    
    %----------------------------------------------------------------------------------------
    %	DOCUMENT INFORMATION
    %----------------------------------------------------------------------------------------
    
    \newcommand{\maintitle}{License Plate Recognizer}
    \newcommand{\course}{Numerical Calculations for Engineering}
    \newcommand{\coursenumber}{ESCUELA TÉCNICA SUPERIOR DE INGENIERÍA Y DISEÑO INDUSTRIAL}
    \newcommand{\class}{International Semester}
    
    %%%%%%%%%%%%%%%%%%%%%%%%%%%%%% HEADER %%%%%%%%%%%%%%%%%%%%%%%%%%%%%%
    \pagestyle{fancy}
    \fancyhf{}
    \fancyhead[LE,RO]{\course}
    \fancyhead[RE,LO]{\maintitle}
    % if working with chapters
    % \fancyfoot[CE,CO]{\leftmark}
    \fancyfoot[LE,RO]{\thepage}
    %%%%%%%%%%%%%%%%%%%%%%%%%%%%%% HEADER %%%%%%%%%%%%%%%%%%%%%%%%%%%%%%
    
    
    %----------------------------------------------------------------------------------------
    %	TITLE PAGE
    %----------------------------------------------------------------------------------------
    \title{\maintitle \\ \course \\{\small\ \coursenumber}} % Title
    
    \author{Pelle \textsc{Reyniers} \\ } % Author name
    
    \date{\today} % Date for the report
    
    \begin{document}
    \sloppy % sloppy is used to enforce that lines are in hbox
    \newgeometry{hmarginratio=1:1}    %% make layout symmetric
    \pagenumbering{gobble}% Remove page numbers (and reset to 1)
    \begin{titlepage}
    \maketitle % Insert the title, author and date
    
    \vfill
    \begin{center}
    \includegraphics[width=0.6\textwidth]{upm_logo.png} %
    \end{center}
    %each \vfill will expand vertically the same amount until the entire page is filled
    \vfill
    \vfill
    \vfill
    
    \begin{center}
    \begin{tabular}{l r}
    Due Date: & 17 May 2019 \\ % Date the experiment was performed
    \\
    Class : & \class \\
    \\
    \end{tabular}
    \end{center}
    \vfill
    \vfill
    \end{titlepage}
    \clearpage
    
    
    %----------------------------------------------------------------------------------------
    %	ROMAN PART OF THE REPORT
    %----------------------------------------------------------------------------------------
    
    \pagenumbering{Roman}
    \section*{Abstract}
    \subfile{sections/abstract.tex}
    \addcontentsline{toc}{section}{Abstract}
    \newpage
    \section*{Introduction}
    \subfile{sections/section0.tex}
    \addcontentsline{toc}{section}{Introduction}
    
    \newpage
    \tableofcontents
    \newpage
    \listoffigures
    \addcontentsline{toc}{section}{List of Figures}
    \lstlistoflistings
    \addcontentsline{toc}{section}{List of Listings}
    \newpage

    %----------------------------------------------------------------------------------------
    %	REAL REPORT
    %----------------------------------------------------------------------------------------
    \pagenumbering{arabic}

    %\section{Problem Introduction}
    \subfile{sections/section1.tex}
    \newpage

    %\section{Problem Breakdown}
    \subfile{sections/section2.tex}
    \newpage

    %\section{Component Overview}
    \subfile{sections/section3.tex}
    \newpage

    %\section{Tests and Results}
    \subfile{sections/section4.tex}
    \newpage

    %\section{Development proces}
    \subfile{sections/section5.tex}
    \newpage

    %\section{Critical Reflection}
    \subfile{sections/section6.tex}
    \newpage

    %\section{Conclusion}
    \subfile{sections/section7.tex}
    \newpage

    %\section{Possible Extensions}
    \subfile{sections/section8.tex}
    \newpage

    %----------------------------------------------------------------------------------------
    %	AFTERMATH
    %----------------------------------------------------------------------------------------
	\appendix
	\bibliographystyle{ieeetr}
    \bibliography{main}
    \addcontentsline{toc}{section}{References}
    \newpage
    %\section{Appendix}
    \subfile{sections/appendix.tex}
    %\subfile{sections/appendix.tex}

    \clearpage
    \end{document}
    